% The future of BlaTeX is in the open source community. Future maintenance will be based on issues reported on Github by actual users. The code will be distributed through Github as well, and users will be able to contribute to its development in the spirit of open-source.

BlaTeX was ultimately an experiment in niche fields. It caters to a (admittedly small) set of people who are either LaTeX enthusiasts, Haskell enthusiasts, or tech-savvy blogging enthusiasts. LaTeX is widely popular in certain fields (specifically math and computer-science), but virtually unknown past that. Haskell attracts fervent users who are passionate about and proud of the code they produce, but is not as popular as more main-stream languages like Java. It may seem like there are a lot of bloggers out there, but only a subset of them are tech-savvy enough to consider a software like this. These factors imply that BlaTeX would not be a very significant project: relevant to only a small population. But just because that population is small does not mean it does not mean it does not exist or can be ignored. Thus the significance of BlaTeX does not lie in its broad reach, but rather in its commitment to solve a very specific problem very well.  

% The specificity of the set of people who would like this project is precisely why it was successful. The significance of BlaTeX does not lie in its broad reach, but rather in its commitment to solve a very specific problem very well. 

%With that being said, the specificity of people who would find this project compelling is why it was successful. 

%The question was whether or not  Although the intersection of these two groups may seem rather insignificant, this project is useful to a group of people, which is significant. 