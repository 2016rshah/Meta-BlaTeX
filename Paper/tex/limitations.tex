  % Equipment, time, materials, etc.

  One potential limitation of this project is that it only compiles links to PDF files. Because LaTeX can't generate HTML, the blog will only be a series of interlinked PDF files rather than a series of interlinked HTML files. This is not necessarily a limitation, but it may interrupt the status quo of typical blogs made of HTML pages. 

  % In order to parse the user's LaTeX files into an abstract syntax tree that can be consumed by Haskell, this project uses a Haskell package called \texttt{HaTeX}. This is a wonderful package (most of the time). The problem arises in a specific parsing instance. In LaTeX, a math environment is defined between two dollar-signs (\texttt{\$MATH GOES HERE\$}). In order to use dollar-signs as actual dollar-signs (like ``You owe me \$5''), you use an escape character before the symbol (\textbackslash). This is an exception to the math environment rule. However, HaTeX is not programmed to catch this exception gracefully. It matches pairs of dollar-signs and treats everything in between as math. If there are an odd number of dollar signs, or only one (even if they are escaped properly) HaTeX will just flat out fail to parse the entire file and thus BlaTeX will similarly ignore the post. 

  Dates are formatted in \texttt{DD MONTH YEAR} which contributes to the minimalist, classy theme of PDF blog posts; this date format is also used in the internal implementation. However, this becomes an issue if multiple posts are authored on same day because they will not be displayed in the correct order. 