  % Explain the steps you took such that your project could be repeated by others (how you did it)

  In general terms: a directory containing posts is examined by BlaTeX, each post is parsed into an abstract syntax tree, information from this tree is parsed into a Haskell Post data structure, each post is stored into a list, the list is sorted by the date, each object is converted into it's corresponding HTML, and this HTML is outputted into an output file at a user-specified location. 

  \subsubsection{The \texttt{Post} Algebraic Data Strucure}

    In Haskell, the equivalent of an ``object'' (used in object oriented programming languages), is an ``algebraic data type (ADT)''. To represent each post, a \texttt{Post} algebraic data type was defined with meta-information for each post. The final definition for the Post was:

    \begin{minted}{haskell}
data Post = Post {
  fileName :: String
  , postTitle :: String
  , postAuthor :: String
  , postDate :: DateTime
  , syntaxTree :: LaTeX
    }
    deriving (Eq, Show)
    \end{minted}

  This defines the structure of a Post based on each attribute it would have and what type that attribute would have.

  \subsubsection{BlazeHTML}
    The first part of the project that I tacked was generating HTML elements that would need to be inserted into the final file. Each post object would generate the following LI element:

  \begin{minted}{html}
<li class='blog-post'>
    <a class='post-link' href='posts/example-post.pdf'>
        Example Post
    </a>
    <div class='post-date'>
        1 January 2015
    </div>
</li>
  \end{minted}

  \subsubsection{HaTeX}
    The second part of the project required that I read in LaTeX files for the posts and use them to extract certain meta-information about each post. This included the title of the post (inserted into the \texttt{.post-link} tag) and the date of the post (used to sort posts and displayed in the \texttt{.post-date} tag). 

  \subsubsection{TagSoup}

    The user supplies a layout file which defines the format and style of their blog. BlaTeX finds the line \texttt{<ul id='blog-posts'></ul>} and inserts all the posts at that point. In order to do so, the project reads in an HTML file and derives it's abstract syntax tree with a package called TagSoup. TagSoup converts an HTML file into a list of HTML elements, and the HTML generated with BlazeHTML is inserted at the specified point. 

  \subsubsection{Dates}

    Support for dates was one of the final undertakings addressed. The Data.Dates package was used to parse the LaTeX dates. This allowed me to sort the list of posts chronologically.  