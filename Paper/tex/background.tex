
My project, called BlaTeX, is a static-site compiler written in Haskell that compiles LaTeX posts into a deploy-able site. It is inspired by a similar Ruby tool called Jekyll that compiles Markdown posts into a static, deploy-able site. However, it allows the user to write posts in LaTeX rather than Markdown (because LaTeX is a more powerful tool) and is overall a more lightweight blogging software. 

% Typically, static-sites like blogs are built with a tool called Jekyll. This Ruby project compiles Markdown posts into a static HTML site. \bx is similar, it is a static-site compiler written in Haskell that compiles LaTeX posts into a deploy-able site. The following description has been edited to highlight the minor differences between Jekyll and \bx:

% \sout{Jekyll} BlaTeX is a simple, blog-aware, static site generator. It takes a template directory containing \sout{raw text files in various formats} LaTeX files\sout{, runs it through a converter (like Markdown) and our Liquid renderer,} and spits out a complete, ready-to-publish static website suitable for serving with your favorite web server. \sout{Jekyll also happens to be the engine behind GitHub Pages, which means} you can use \sout{Jekyll} BlaTeX to host your project's page, blog, or website from GitHub's servers for free.

Another popular content-management system for blogging is Wordpress. However, Wordpress posts are edited in the Wordpress Administration Panel as ``What you see is what you get''. Neither Markdown nor LaTeX are ``What you see is what you get'', and thus Wordpress caters to a different audience (a group of mostly non-technical authors). 