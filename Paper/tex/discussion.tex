BlaTeX benefits mathemeticians, computer-scientists, and students of both disciplines who would like to maintain a technical blog about their work. It allows them to write their posts with the same advanced tools they use for their professional work. 

The project is also a milestone for the language used to write it. Often the popularity of computer science languages is based heavily on the amount of open-source projects implemented with that language. Haskell is often disparaged because it is rarely used for personal projects, however the Haskell community is trying to refute this claim. Thus, BlaTeX contributes to the growing population of popular projects implemented in Haskell and demonstrates that Haskell is a reasonable choice for future open-source projects. 

LaTeX is widely constrained to only being used in large-scale academic or professional projects, which is why it is such a mature piece of software. However BlaTeX broadens the scope of LaTeX itself by allowing people to utilize the strenght of LaTeX on a day-to-day basis. 

Future work can concentrate on the limitations outlined above, and specifically the fact that as of right now all posts are PDFs. Pandoc is a tool that converts between file formats (and is also written in Haskell). In order to address the potential limitation that BlaTeX doesn't produce HTML posts, further extensions could explore Pandoc conversions to HTML prior to static-site compilation. 